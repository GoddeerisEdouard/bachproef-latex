%%=============================================================================
%% Methodologie
%%=============================================================================

\chapter{\IfLanguageName{dutch}{Methodologie}{Methodology}}
\label{ch:methodologie}

%% TODO: Hoe ben je te werk gegaan? Verdeel je onderzoek in grote fasen, en
%% licht in elke fase toe welke stappen je gevolgd hebt. Verantwoord waarom je
%% op deze manier te werk gegaan bent. Je moet kunnen aantonen dat je de best
%% mogelijke manier toegepast hebt om een antwoord te vinden op de
%% onderzoeksvraag.

In dit onderdeel wordt de methodologie van data mining en webscraping applicaties binnen de InsurTech geïllustreerd met gedetailleerde beschrijvingen en discussies omtrent iedere fase.

\section{Opbouw van het onderzoek}
Voor dit onderzoek zal gewerkt worden met de vooraf omschreven tools en technieken.
Data die online verzameld wordt zal verrijkt worden door logischerwijs irrelevante info eruit te filteren en te combineren met gelijksoortige informatie.
De eerste stap om het onderzoek te kunnen voeren is data verzamelen en verwerken die gebruikt kan worden om 
We zullen een framework gebruiken om ons webscraperproces te vergemakkelijken. Het framework dat als basis zal dienen voor de web scraper moet gekozen worden voordat het scrapen kan beginnen. Selenium en Puppeteer zijn twee verschillende, eenvoudige frameworks die kunnen worden gebruikt om gegevens van een website te scrapen.

Selenium is het bekendere framework, hoewel dit framework eerder gebruikt wordt om webbrowser testen te automatiseren. Door hiervan gebruik te maken zal het voor dit onderzoek in de toekomst makkelijk zijn grote tabellen data uit een webpagina op te slaan in een databank om dan elders te gaan herbebruiken, meerbepaald in het prototype. Het is een feit dat Selenium een webbrowser opent en visueel de taken nabootst die een gebruiker zou uitvoeren om een webpagina te testen. Dit kan voor sommige doeleinden, zoals het minimaliseren van werkgeheugen voor de computer een factor zijn om dit framework te mijden. Het is wel mogelijk deze "functie" uit te schakelen en zonder visualisatie te werken, wat ook wel "headless browsing" genoemd wordt. Dit is niet nodig, omdat efficiëntie voor het ophalen van data voor ons prototype niet echt een rol speelt. 

Het andere framework genaamd Puppeteer heeft dan weer enkel een headless mode, waarbij dus geen webbrowser visueel voor de ontwikkelaar verschijnt. 