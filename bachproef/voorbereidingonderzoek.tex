%%=============================================================================
%% Voorbereiding Onderzoek
%%=============================================================================

\chapter{\IfLanguageName{dutch}{Voorbereiding van het Onderzoek}{}}
\label{ch:voorbereiding-onderzoek}

\section{\IfLanguageName{dutch}{Inleiding}{}}
\label{sec:inleiding}

Dit deel van het onderzoek gaat op zoek naar onderzoekstechnieken om in het volgende hoofdstuk uit te werken en te implementeren.
Men gaat eerst de verschillende soorten verzekeringen oplijsten om dan een onderscheid te kunnen maken tussen specialisaties.
Met die kennis bij de hand wordt er op zoek gegaan naar bedrijven of websites die technologische tools in de verzekeringswereld aanbieden. Wanneer deze gevonden zijn, kijkt men bij welke categorie deze toepassingen precies horen. Dit wordt mooi opgelijst. 
Met deze kennis bij de hand gaat men ten slotte op zoek naar waar zich nog gaten bevinden ofwel waar verbetering in gebracht kan worden. Men lijst ook al meteen op waarom dit nuttig kan zijn en voor wie.

\section{\IfLanguageName{dutch}{Bestaande toepassingen}{}}
\label{sec:bestaande-toepassingen}

De Insurtech heeft een uitgebreid aanbod ban mogelijke toepassingen. Dit soort toepassingen kunnen zich in verschillende stadia of categorieën van het verzekeringsproces bevinden. Zoals het digitale bijstaan van een klant of digitale hulp kunnen bieden tijdens het opstellen van een verzekeringscontract. Specifiek kan een toepassing gaan over waarschuwen bij mogelijks onweer via API’s. Daar is <bedrijfsnaam> een voorbeeld van. Andere reeds geïmplementeerde insurtech toepassingen zijn <te onderzoeken>
Het valt op dat voornamelijk toepassingen bestaan voor huisverzekeringe
Wat opvallend weinig aan bod komt, aijn toepassingen voor autoverzekering. Bij dit verzekeringsproces...
