%%=============================================================================
%% Voorbereiding Onderzoek
%%=============================================================================

\chapter{\IfLanguageName{dutch}{Voorbereiding van het Onderzoek}{}}
\label{ch:voorbereiding-onderzoek}

De Insurance Technology is een vooruitstrevende technologie die in de verzekeringswereld gebruikt wordt. Verzekeringsmaatschappijen willen voortdurend op kosten en tijd besparen, dit is net waar de InsurTech zich op focust. Eerst en vooral werd onderzocht worden hoe een standaard verzekeringsproces precies verloopt en welke documenten hiervoor ingevuld moeten worden. Vervolgens werd gezocht naar welke stadia in dit proces geautomatiseerd of versneld kunnen worden. Dan zal voor deze onderdelen een automatisatie gezocht worden die men verwerft via data die men via webscraping of het gebruik van API’s op externe webpagina’s verkrijgt. Het onderzoek zou een oplossing moeten bieden voor het lange “invulproces” die de klant en de makelaar moeten doorlopen om een verzekeringscontract op punt te zetten. De verzekeringswereld is nog vrij conservatief, vele contracten vergen veel contact met de klant. In de toekomst zal dit veranderen, het onderzoek zou een steentje moeten bijdragen aan de mogelijke methodes waarin dit proces kan versneld worden. tools zoals Selenium en Puppeteer gebruikt worden. Bij het gebruik API’s maakt dit onderzoek gebruik van de requests library in Python indien webscraping niet nodig of mogelijk is. Verder zal voor het onderzoeken van externe API’s de Google Developer Tools een hulp bieden in het uitwerken van de toepassingen. Als eindresultaat zal een minimalistische simulatie aantonen dat de toepassing wel degelijk innovatief is.

\section{\IfLanguageName{dutch}{Inleiding}{}}
\label{sec:inleiding}

Dit deel van het onderzoek gaat op zoek naar onderzoekstechnieken om in het volgende hoofdstuk uit te werken en te implementeren.
Men gaat eerst de verschillende soorten verzekeringen oplijsten om dan een onderscheid te kunnen maken tussen specialisaties.
Met die kennis bij de hand wordt er op zoek gegaan naar bedrijven of websites die technologische tools in de verzekeringswereld aanbieden. Wanneer deze gevonden zijn, kijkt men bij welke categorie deze toepassingen precies horen. Dit wordt mooi opgelijst. 
Met deze kennis bij de hand gaat men ten slotte op zoek naar waar zich nog gaten bevinden ofwel waar verbetering in gebracht kan worden. Men lijst ook al meteen op waarom dit nuttig kan zijn en voor wie.

\section{\IfLanguageName{dutch}{Bestaande toepassingen}{}}
\label{sec:bestaande-toepassingen}

De Insurtech heeft een uitgebreid aanbod ban mogelijke toepassingen. Dit soort toepassingen kunnen zich in verschillende stadia of categorieën van het verzekeringsproces bevinden. Zoals het digitale bijstaan van een klant of digitale hulp kunnen bieden tijdens het opstellen van een verzekeringscontract. Specifiek kan een toepassing gaan over waarschuwen bij mogelijks onweer via API’s. Daar is <bedrijfsnaam> een voorbeeld van. Andere reeds geïmplementeerde insurtech toepassingen zijn <te onderzoeken>
Het valt op dat voornamelijk toepassingen bestaan voor huisverzekeringe
Wat opvallend weinig aan bod komt, aijn toepassingen voor autoverzekering. Bij dit verzekeringsproces...
