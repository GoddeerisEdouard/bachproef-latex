%%=============================================================================
%% Voorbereiding Onderzoek
%%=============================================================================

\chapter{\IfLanguageName{dutch}{Voorbereiding op het Onderzoek}{}}
\label{ch:voorbereiding-onderzoek}

\section{\IfLanguageName{dutch}{Inleiding}{}}
\label{sec:inleiding}

Dit deel van het onderzoek gaat op zoek naar onderzoekstechnieken om in het volgende hoofdstuk uit te werken en te implementeren.
Allereerst werden de verschillende soorten verzekeringen opgelijst om dan een onderscheid te kunnen maken tussen specialisaties.
Met die kennis bij de hand werd vervolgens op zoek gegaan naar bedrijven of websites die technologische tools en services in de verzekeringswereld aanbieden. Wanneer deze gevonden werden, kon iedere gevonden service of tools geplaatst worden in een bijhorende categorie. Dit werd opgelijst in een tabel.

Met deze kennis bij de hand gaat men ten slotte op zoek naar waar zich nog gaten bevinden ofwel waar verbetering in gebracht kan worden. Men lijst ook al meteen op wie baat heeft bij deze categorie en er worden voorbeelden gegeven.

\section{Nood aan data}
Met behulp van Python werd met enkele lijnen code via Selenium Webdriver data opgehaald van webpagina's om in het prototype te gebruiken en verwerken.
Eén van de doelen van InsurTech is namelijk ook grote hoeveelheden aan data verwerken, waar de webscraper aan te pas komt. Het toepassen van methoden om deze data te achterhalen worden hierbij ook detailleerd uitgelegd zodat dit begrijpbaar is voor een lezer met weinig tot geen voorkennis.

\section{\IfLanguageName{dutch}{Bestaande toepassingen}{}}
\label{sec:bestaande-toepassingen}
De Insurtech heeft een uitgebreid aanbod aan mogelijke toepassingen. Dit soort toepassingen kunnen zich in verschillende stadia of categorieën van het verzekeringsproces bevinden. Zoals het digitale bijstaan van een klant of digitale hulp kunnen bieden tijdens het opstellen van een verzekeringscontract. Verder omvat het andere zaken zoals smart phone applicaties, consumenten wearables, claim afhandeling tools, online beleidsbeheer en geautomatiseerde verwerking. Met deze technologieën kunnen verzekeringsmaatschappijen gegevens van consumenten verzamelen en analyseren, zodat ze zich kunnen richten op de juiste klanten tegen de juiste prijs, hen kunnen aanmoedigen zich minder gevaarlijk te gedragen en de kosten van claims kunnen minimaliseren. Natuurlijk hebben verzekeraars altijd al informatie over beleidsbeheer geanalyseerd, maar in dit tijdperk van big data proberen ze nieuwe informatiebronnen aan te boren, zoals sociale media, en geavanceerde technologie te gebruiken om al deze gegevens te verwerken, met behulp van machine learning en kunstmatige intelligentie om betere voorspellingen te doen en deze kennis toe te passen om de besluitvorming en bedrijfsplanning te verbeteren. Een andere is het opzetten van autonome digitale verzekeringsagenten zijnde chat bots. Deze maken gebruik van geïndividualiseerde gebruikersgegevens om gesprekken met consumenten op maat te maken. Andere bedrijven die applicaties hebben ontworpen die verzekeringen op uurbasis aanbieden, ideaal als een klant het voertuig van een vriend wil lenen voor de middag. Dan is er nog Blockchain, die doeltreffend kan zijn in de strijd tegen diefstal en verzekeringsfraude. Een blockchain is een moeilijk idee, maar het is gewoon een peer-to-peer. Een openbaar gedistribueerd grootboek is een gedecentraliseerde registratie van informatie die beschikbaar is voor verschillende operatoren. Het maakt de oprichting van een fraudebestendige database mogelijk die de eigendom en overdracht van activa kan traceren. Verzekeraars zijn voortdurend bereid gegevens uit te wisselen om kosten te besparen en fraude tegen te gaan.
Wat betreft reeds geïmplementeerde InsurTech toepassingen zijn, is deel van het onderzoek dat volgt.
