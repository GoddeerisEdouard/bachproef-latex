%%=============================================================================
%% Conclusie
%%=============================================================================

\chapter{Conclusie}
\label{ch:conclusie}

% TODO: Trek een duidelijke conclusie, in de vorm van een antwoord op de
% onderzoeksvra(a)g(en). Wat was jouw bijdrage aan het onderzoeksdomein en
% hoe biedt dit meerwaarde aan het vakgebied/doelgroep? 
% Reflecteer kritisch over het resultaat. In Engelse teksten wordt deze sectie
% ``Discussion'' genoemd. Had je deze uitkomst verwacht? Zijn er zaken die nog
% niet duidelijk zijn?
% Heeft het onderzoek geleid tot nieuwe vragen die uitnodigen tot verder 
%onderzoek?

Door basisregels voor gegevensvalidatie in te stellen, kan een bedrijf georganiseerde normen aanhouden die het werken met gegevens efficiënter maken.
De gegevensvalidatie in combinatie met opgehaalde data op het web door middel van een webscraper kan er ook voor zorgen dat bepaalde informatie overbodig wordt om in te vullen, wat opnieuw voor tijdswinst voor de klant zorgt. Dit zijn exact de resultaten waarop voordien gehoopt werd en sluit aan bij de hypothese. Bedrijven zouden met behulp van dit onderzoek moeten inzien dat hun bedrijf voordeel kan halen uit het toepassen van webscraping. "Less is more" wordt ook wel gezegd over architectuur, maar ook dit is van toepassing bij het opvragen van data bij een klant: hoe simpeler, hoe beter. Hoewel er minder data van een klant wordt gevraagd, ontvangt het bedrijf evenveel informatie, het minimalistisch houden van deze velden, maakt het aantrekkelijker en gebruiksvriendelijker voor de eindgebruiker. Het wordt aanbevolen af te staooeb van fysieke kopieën van papierwerk en in plaats daarvan gebruik van online formulieren en digitale handtekeningen om het proces eenvoudig en toegankelijk te maken. Het ingeven van een automerk en model bij een autoverzekering een voorbeeld van. Deze stappen die uitgevoerd worden op het invullen digitale autoverzekeringsformulieren kunnen uitgebreid worden naar andere verzekeringsformulieren die gelijkaardige optimalisatie kunnen bieden. Er is echter een deel van het onderzoek binnen het proces nog onduidelijk. Namelijk: "welke analyse kan toegepast worden op de informatie die doorgegeven wordt"? Hier kunnen gedachten de vrije baan opgaan en een eigen interpretatie geven. Zoals al dan niet een omniumverzekering aanbevelen op basis van de ingevulde data. Dit sluit dan minder aan op webscraping. Deze vraag nodigt uit voor een verder onderzoek waar de data precies gebruikt kan worden.

