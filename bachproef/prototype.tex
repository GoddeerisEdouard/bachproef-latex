%%=============================================================================
%% Prototype
%%=============================================================================
\chapter{\IfLanguageName{dutch}{Prototype}{}}
\label{ch:prototype}

\section{\IfLanguageName{dutch}{Inleiding}{}}
\label{sec:inleiding}

In dit hoofdstuk worden de bepaalde technieken van het voorgaande hoofdstuk uitgewerkt op een toepassing met behulp van Python en webscraping. Op basis van voorgaand besproken datamining technieken en web scraping toepassingen, wordt een prototype opgesteld die een nuttige meerwaarde biedt in de Insurtech.

\section{\IfLanguageName{dutch}{Gebruikte tools}{}}
\label{sec:Gebruikte tools}

Om het prototype op te stellen wordt eerst en vooral een set van tools opgelijst die binnen de werkomgeving gebruikt worden. Zo wordt er gekozen voor de programmeertaal Python.
Bij webscraping is het belangrijk om de juiste tool te vinden die voldoet aan de benodigdheden binnen het project. Zo is het bij dit onderzoek de bedoeling enkel webscrapers te gebruiken die draaien op de programmeertaal Python. In de vooraf aangehaald in de Stand van zaken wordt gebruik gemaakt van een veel gebruikte webscraping tool die voldoet aan de eisen die voor dit onderzoek nodig zijn.
% TODO mogelijks hier extra aanvullen waarom we voor Selenium kiezen

\section{Toepassingen}
% TODO voeg inleiding toe

\subsection{Bestaande toepassingen}

Om een nieuwe toepassing te ontwikkelen die InsurTech gerelateerd is, is het van belang op zoek te gaan naar bestaande toepassingen. Wegens dat toepassingen binnen de InsurTech op globaal niveau een brede waaier is om onderzoek op te doen, focust dit onderzoek zich eerder op de bestaande toepassingen in België. Er wordt op zoek gegaan naar een selectie van InsurTech bedrijven in België die publiekelijk online informatie verschaffen over de services die zij bieden. 

Wanneer op het web gezocht wordt naar Belgische InsurTech bedrijven is een prominent resultaat \href{https://www.fintechbelgium.be/}{Fintech Belgium}. Deze gemeenschap, opgericht in 2015, wil zo veel mogelijk FinTech bedrijven, dus bedrijven (in)direct gerelateerd met de financiële service industrie, betrokken krijgen. Het doel van deze gemeenschap is de functionaliteit binnen de industrie te verbeteren alsook bestaande problemen aan te pakken door middel van hun connecties binnen en buiten het opgebouwde netwerk. Eén van de leden binnen het bedrijf is WeGroup, het InsurTechbedrijf dat reeds kort vermeld is in het Voorwoord. Voornamelijk worden de verzekeringstypes die het bedrijf dekt onderverdeeld in vier categorieën ofwel types. Zijnde autoverzekering, woonverzekering, familiale verzekering en rechtsbijstandverzekering.
Het is van belang dat banken die verzekeringen bieden vaak een simulatie van een verzekeringscontract aan om klanten de kost van hun verzekering te laten inschatten en deze mogelijks te kunnen vergelijken.

\subsubsection{Autoverzekering stappen}
Zo is er voor de autoverzekering bij AXA: \href{https://www.fo.axa.be/eauto/risk?dsfns=customers.be.axa.retail.mobility.contract.new.auto&LANG=nl}{AXA autoverzekering}, ING: (\href{https://www.ing.be/nl/retail/insurance/vehicles/car-insurance}{ING autoverzekering}), KBC: (\href{https://www.kbcbrussels.be/retail/en/processes/vehicle/autoverzekering-simuleren.html}{KBC autoverzekering}), Baloise: \href{https://www.berekenjeautopremie.be/nl/Baloise/l/met-welke-wagen-rijdt-u}{Baloise autoverzekering} en vele andere banken / verzekeringsmaatschappijen een webpagina voorzien.
Hierbij wordt eerst gekeken naar de nodige informatie per verzekeringstype.
Wanneer wordt gekeken naar de indeling van autoverzekering wordt aan de klant een gelijkaardige indeling om van gegevens invullen om aan data van een klant te geraken en vervolgens een prijsberekening te doen.
 
\begin{enumerate}[label=Stap \arabic*:]
	\item Selecteer als het voertuig al dan niet ouder dan 2 jaar is
	\item Vul mogelijks het chassisnummer in, indien niet mogelijk, vul info over auto in
	info over auto: 
	\begin{enumerate}
		\item Brandstoftype
		\item Eerste inschrijving
		\item Merk
		\item Model
		\item Type
	\end{enumerate}
	\item Eénmaal dit allemaal ingevuld is wordt de cataloguswaarde (exclusief BTW) van het voertuig gevraagd.
	Dit is de oorspronkelijke waarde van de auto exclusief btw, kortingen, opties en accessoires.
\end{enumerate}
Resultaat: eindbedrag voor autoverzekering

\subsubsection{Woonverzekering stappen}
\emph{Voor eigenaars}

\begin{enumerate}[label=Stap \arabic*:]
	\item selecteer indien u de eigenaar of huurder bent
	\item vul het adres in van de te verzekeren woning
	\item selecteer type woning
	\item selecteer type huis of appartement
	\item invullen extra info over huis of appartement
	\item kiezen hoofdverblijf, verhuren, tweede verblijf, leegstaand
	\item invullen aantal kamers
	\item invullen andere ruimtes
	\item bouwjaar
	\item schade overstromingen afgelopen 5 jaar
	\item selecteer welke schade of kosten wil je gedekt zijn
\end{enumerate}      
Resultaat: eindbedrag voor de woonverzekering

\emph{Voor huurders}

\begin{enumerate}[label=Stap \arabic*:]
	\item selecteer indien u de eigenaar of huurder bent
	\item vul het adres in van de te verzekeren woning
	\item selecteer type woning (rij - open - halfopen)
	\item selecteer type huis of appartement
	\item vul maandelijkse huur in
	\item vul meer informatie in over de woning ( kamers, andere ruimtes, bouwjaar, ... )
	\item selecteer welke schade of kosten wil je gedekt zijn
\end{enumerate}
Resultaat: eindbedrag voor de woonverzekering

\subsubsection{Rechtsbijstand stappen}
% TODO vul in


\subsubsection{Familiale verzekering stappen}
Opnieuw is er voor Belgische verzekeringsmaatschappijen invulformulier met data nodig waardoor een inschatting van kosten en dekking kan worden gemaakt.
Hierbij wordt enkel de vraag gesteld als u verzekerd bent.


\subsection{Replicatie toepassing}
% TODO maak een replicatie van één van de invulprocessen voor het verkrijgen van een verzekeringscontract
Om een toepassing van de InsurTech beter te begrijpen betreft een deel van het onderzoek het repliceren van het digitale invulproces van een verzekerings prijsberekening.
In dit voorbeeld zal de autoverzekering worden gerepliceerd.
Een opmerking is dat het onderzoek geen toegang heeft tot prijsberekeningen bij dit soort data, dus hier wordt voornamelijk gefocust op het efficiënt vragen van data aan de klant.

Wanneer men de stappen van autoverzekering herziet wordt het merk en model gevraagd van de auto.
Om alle verschillende soorten merken en modellen op te vragen is nood aan een gevulde databank.
Online zijn verschillende websites beschikbaar die alle automerken en hun types opslaat.
\href{https://www.car.info/en-se/brands}{car info} is daar één van.
Er valt voornamelijk een lijst van auto's alsook het bouwjaar en dergelijke afbeeldingen op deze website te vinden.

Het eerste wat gedaan moet worden op deze website is dus de data, zijnde alle automerken en hun modellen opslaan.
Dit kan met behulp van de webscraper van Selenium in Python.
In enkele lijnen code wordt per automerk gezocht naar de link die meer details biedt voor dit merk.
Deze link leidt ons door naar alle info over dit merk, waardoor men de bestaande modellen en het bouwjaar kan opgeslagen worden.
Na het ophalen van al deze data wordt dit opgeslagen in een databank, die later dankzij een zoekfunctie snel data kan verschaffen voor de aangevraagde zoektermen.
Vragen zoals "Welke modellen bestaan er voor het merk Ferrari?" of "Welke auto's hebben geen gekend bouwjaar?".
% TODO voeg afbeelding toe die opgehaalde data mooi uitmapt

Nadat de data is opgeslagen en vlot toegankelijk is, wordt het invulformulier nagemaakt.
Wegens dat deze replicatie meer gefocust op de functionaliteit van de toepassing dan naar de lay-out.
Het is ook van belang dat dit formulier CORRECT ingevuld wordt en geen onzin kan meegegeven worden.
Zoals een jaartal van eerste inschrijving meegeven dat in de toekomst ligt.

% TODO voeg afbeelding toe van invulformulier met korte lijst bestaande modellen (zodat dit op een tamelijk kleine screenshot past)
% deze afbeelding moet ook aantonen dat het aan validatie doet

Er valt duidelijk te zien dat dankzij dit digitaal formulier een aantal factoren die voor verwarring of ongeldigheid zouden kunnen zorgen bij ontvangst van het verzekeringskantoor worden geëlimineerd.
Zo is er strikte validatie op de velden, en niet alleen op velden die vrij in te vullen zijn door de eindgebruiker.
Zo worden ook bepaalde velden automatisch ingevuld, of wordt er een selectie aan mogelijke antwoorden gepresenteerd.

In het kort:
door basisregels voor gegevensvalidatie in te stellen, kan een bedrijf georganiseerde normen aanhouden die het werken met gegevens efficiënter maken.
De gegevensvalidatie op zich kan er ook voor zorgen dat bepaalde informatie overbodig wordt om in te vullen, wat opnieuw voor tijdswinst voor de klant zorgt, daar is het chassisnummer een voorbeeld van.


\subsection{Klantenervaring verbeteren}
% TODO vind andere toepassingen van insurtech



\subsection{Data extractie}
Met behulp van Python kunnen we met enkele lijnen code via Selenium Webdriver data ophalen van webpagina's om in het prototype te gebruiken en verwerken.
Eén van de doelen van InsurTech is namelijk ook grote hoeveelheden aan data verwerken, waar de webscraper aan te pas komt.


