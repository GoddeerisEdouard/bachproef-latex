%%=============================================================================
%% Prototype
%%=============================================================================

\chapter{\IfLanguageName{dutch}{Prototype}{}}
\label{ch:prototype}

\section{\IfLanguageName{dutch}{Inleiding}{}}
\label{sec:inleiding}

In dit hoofdstuk worden de bepaalde technieken van het voorgaande hoofdstuk uitgewerkt op een toepassing met behulp van Python en webscraping. Op basis van voorgaand besproken datamining technieken en web scraping toepassingen, wordt een prototype opgesteld die een nuttige meerwaarde biedt in de Insurtech.

\section{\IfLanguageName{dutch}{Gebruikte tools}{}}
\label{sec:Gebruikte tools}

Om het prototype op te stellen wordt eerst en vooral een set van tools opgelijst die binnen de werkomgeving gebruikt worden. Zo wordt er gekozen voor de programmeertaal Python.
Bij webscraping is het belangrijk om de juiste tool te vinden die voldoet aan de benodigdheden binnen het project. Zo is het bij dit onderzoek de bedoeling enkel webscrapers te gebruiken die draaien op de programmeertaal Python, dus is Apache Nutch bijvoorbeeld al niet van toepassing. Zo zijn er nog vele andere webscraping tools die \textcite{Sirisuriya2015} aanhaalt in haar studie. Bij een recent onderzoek naar de meestgebruikte tools in Python kwamen Selenium en Puppeteer als één van de populairste uit de bus. \autocite{Saurkar2018} Vandaar zal ook bij dit onderzoek deze tools gebruikt worden. Deze maken gebruik van webautomatie om hun taak te vervullen.
