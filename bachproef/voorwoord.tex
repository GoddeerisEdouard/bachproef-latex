%%=============================================================================
%% Voorwoord
%%=============================================================================

\chapter*{\IfLanguageName{dutch}{Woord vooraf}{Preface}}
\label{ch:voorwoord}

In het derde jaar Toegepaste Informatica wordt aan de Hogeschool Gent als eindwerk een scriptie verwacht, relevant aan de opleiding. Als projecten buiten school ben ik al altijd gefascineerd geweest door automatisatie. Dit gaat dan over autoclickers op computerspellen of het opsommen van data zonder te moeten manueel kopiëren en plakken. In het kort, herhaaldelijke taken uitsluiten door efficiënter een computer te gebruiken. In mijn vrije tijd heb ik vaak al kleine projecten uitgeschreven om te achterhalen hoe ik telkens kleine zaken kan automatiseren. Ook als specialisatie Data Engineer en AI leek het zeer gepast een aansluiten onderzoek hierover te doen. Tijdens het tweede semester werd van mij verwacht dat ik stage zou lopen bij een informatica bedrijf. Ik ging op zoek in Gent naar een bedrijf die aan mijn interesses voldeed. Zo kwam ik bij WeGroup terecht, een startup InsurTech bedrijf die een verzekeringstool aanbiedt voor verzekeringsmakelaars. Tijdens mijn eerste week stage had ik al snel het idee om onderzoek te doen die een combinatie is van de branche waarin mijn stagebedrijf zit en mijn passie voor data en automatisatie. Zo kwam ik bij webscraping en de toepassingen ervan binnen de InsurTech.

Deze bachelorproef had ik niet alleen kunnen uitwerken. Daarom bedank ik graag mijn stagebegeleider bij WeGroup Freek Verschelden die me duidelijkheid bracht over InsurTech en de tools die binnen het bedrijf gebruikt worden of al geïmplementeerd waren.
Ook bedank ik graag mijn co-promotor Benjamin Vertonghen die me concreet feedback en inspiratie gaf bij waar ik precies met mijn onderzoek naartoe wou.

