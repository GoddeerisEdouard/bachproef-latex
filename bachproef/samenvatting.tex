%%=============================================================================
%% Samenvatting
%%=============================================================================

% TODO: De "abstract" of samenvatting is een kernachtige (~ 1 blz. voor een
% thesis) synthese van het document.
%
% Deze aspecten moeten zeker aan bod komen:
% - Context: waarom is dit werk belangrijk?
% - Nood: waarom moest dit onderzocht worden?
% - Taak: wat heb je precies gedaan?
% - Object: wat staat in dit document geschreven?
% - Resultaat: wat was het resultaat?
% - Conclusie: wat is/zijn de belangrijkste conclusie(s)?
% - Perspectief: blijven er nog vragen open die in de toekomst nog kunnen
%    onderzocht worden? Wat is een mogelijk vervolg voor jouw onderzoek?
%
% LET OP! Een samenvatting is GEEN voorwoord!

%%---------- Nederlandse samenvatting -----------------------------------------
%
% TODO: Als je je bachelorproef in het Engels schrijft, moet je eerst een
% Nederlandse samenvatting invoegen. Haal daarvoor onderstaande code uit
% commentaar.
% Wie zijn bachelorproef in het Nederlands schrijft, kan dit negeren, de inhoud
% wordt niet in het document ingevoegd.

\IfLanguageName{english}{%
\selectlanguage{dutch}
\chapter*{Samenvatting}
\lipsum[1-4]
\selectlanguage{english}
}{}

%%---------- Samenvatting -----------------------------------------------------
% De samenvatting in de hoofdtaal van het document

\chapter*{\IfLanguageName{dutch}{Samenvatting}{Abstract}}

In dit onderzoek zal via webscraping gezocht worden naar toepassingen voor InsurTech, meerbepaald Insurance Technology bedrijven om hun klanten een zo gebruiksvriendelijk mogelijke ervaring te bieden. De digitalisatie of automatisatie van het invullen van verzekeringsdocumenten kan een verzekeringsmakelaar veel tijd besparen die elders nuttig kan besteed worden. Python is de ideale programmeertaal naar keuze om data analyse en wetenschappelijk onderzoek te doen. Tools om te webscraping zoals Selenium of Puppeteer en het gebruik van publieke websites en Google Dev Tools zullen hierbij een grote hulp bieden tijdens het onderzoek. Webscrapers kunnen webpagina’s uitlezen om zo automatisch data te indexeren en te verwerken. Het is immers niet altijd nodig webscraping toe te passen wanneer websites een API bieden, wat het voor ontwikkelaars makkelijker maakt hun diensten te gebruiken. Door middel van een prototype op te stellen voor een specifieke InsurTech categorie wordt onderzocht als er mogelijkheden zijn die de kosten en tijd van een verzekeringsmaatschappij uitsparen. Uit resultaten van het onderzoek blijkt dat via webscraping meerdere hulpzame toepassingen zijn die de verzekeringsmaatschappij kan gebruiken. Dit statement wordt bevestigd door de resultaten en optimalisatie die het prototype te bieden heeft voor een verzekeringsproces. Uit de resultaten blijkt dat een verzkeringscategorie, meerbepaald het afsluiten van verzekeringscontracten, geoptimaliseerd kan worden. In detail kan via webscraping data gebruikt worden die ervoor zorgt dat invulvelden voor de klant weggelaten kunnen worden. Niet alleen kunnen deze velden weggelaten worden, maar door validatie hierop kunnen kosten bespaard worden door werknemers die controleren als de ingevulde formulieren wel kloppen weg te laten. Wegens dat InsurTech nog relatief nieuw is en makelaars steeds op zoek zijn naar nieuwe innovatie binnen hun bedrijf, is dit zeker een onderzoek dat in de toekomst nóg relevanter zal worden.
