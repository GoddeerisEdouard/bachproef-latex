%%=============================================================================
%% Samenvatting
%%=============================================================================

% TODO: De "abstract" of samenvatting is een kernachtige (~ 1 blz. voor een
% thesis) synthese van het document.
%
% Deze aspecten moeten zeker aan bod komen:
% - Context: waarom is dit werk belangrijk?
% - Nood: waarom moest dit onderzocht worden?
% - Taak: wat heb je precies gedaan?
% - Object: wat staat in dit document geschreven?
% - Resultaat: wat was het resultaat?
% - Conclusie: wat is/zijn de belangrijkste conclusie(s)?
% - Perspectief: blijven er nog vragen open die in de toekomst nog kunnen
%    onderzocht worden? Wat is een mogelijk vervolg voor jouw onderzoek?
%
% LET OP! Een samenvatting is GEEN voorwoord!

%%---------- Nederlandse samenvatting -----------------------------------------
%
% TODO: Als je je bachelorproef in het Engels schrijft, moet je eerst een
% Nederlandse samenvatting invoegen. Haal daarvoor onderstaande code uit
% commentaar.
% Wie zijn bachelorproef in het Nederlands schrijft, kan dit negeren, de inhoud
% wordt niet in het document ingevoegd.

\IfLanguageName{english}{%
\selectlanguage{dutch}
\chapter*{Samenvatting}
\lipsum[1-4]
\selectlanguage{english}
}{}

%%---------- Samenvatting -----------------------------------------------------
% De samenvatting in de hoofdtaal van het document

\chapter*{\IfLanguageName{dutch}{Samenvatting}{Abstract}}

In dit onderzoek zal via webscraping en datamining gezocht worden naar toepassingen voor InsurTech, meerbepaald Insurance Technology bedrijven om hun klanten een zo gebruiksvriendelijk mogelijke ervaring te bieden. De digitalisatie of automatisatie van het invullen van verzekeringsdocumenten kan een verzekeringsmakelaar veel tijd besparen die elders nuttig kan besteed worden. Tools om te webscraping zoals Selenium of Puppeteer en het gebruik van API’s zullen hierbij een grote hulp bieden tijdens het onderzoek. Webscrapers kunnen webpagina’s uitlezen om zo automatisch data te indexeren en te verwerken. Het is immers niet altijd nodig webscraping toe te passen wanneer websites een API bieden, wat het voor ontwikkelaars makkelijker maakt hun diensten te gebruiken. Voor het gebruik van API’s zal via de Google Developer Tools gekeken worden naar in-en uitgaande API requests. Uit resultaten van het onderzoek zou moeten blijken dat via webscraping en datamining er zeker meerdere hulpzame toepassingen zijn die de verzekeringsmaatschappij kan gebruiken. Dit wegens dat InsurTech nog relatief nieuw is en er nog vele conservatieve makelaars zijn, is dit zeker een onderzoek dat in de toekomst nóg relevanter zal worden.
